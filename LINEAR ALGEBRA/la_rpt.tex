\documentclass{article}
\usepackage[utf8]{inputenc}
\usepackage{amsmath} 
\usepackage{geometry}
\title{LA report}
\author{rishabhmalik26 }
\date{November 2020}

\begin{document}


\maketitle

\section{Linear Codes}
In this chapter, we study linear error-control codes, which are special codes with rich mathematical
structure. Linear codes are widely used in practice for a number of reasons. One reason is that they
are easy to construct. Another reason is that encoding linear codes is very quick and easy. Decoding
is also often facilitated by the linearity of a code. The theory of general linear codes requires the
use of abstract algebra and linear algebra, but we will begin with some simpler binary linear codes

\subsection{Binary Codes}
A binary linear code C of length n is a set of binary n-tuples such that the
component wise modulo 2 sum of any two codewords is contained in C.

\textbf{Example 1.1.1.} The set {0000, 0101, 1010, 1111} is a binary linear code, since the sum of any two
codewords lies in this set. Notice that this is a (4, 4, 2) code because the codewords have length 4,
there are 4 codewords, and the minimum distance between codewords is 2.

\textbf{Example 1.1.2.} The set {0000000, 0001111, 0010110, 0011001, 0100101, 0101010, 0110011, 0111100,
1000011, 1001100, 1010101, 1011010, 1100110, 1101001, 1110000, 1111111} is a binary linear code
known as a Hamming code. Notice that this is a (7, 16, 3) code because the codewords have length
7, there are 16 codewords, and the minimum distance is 3.

\section{Fields, Vector Spaces, Subspaces and General Linear Codes}
In order to develop the theory for general linear codes, we need some abstract and linear algebra.
Learning the definitions and getting comfortable with the examples will be very important.
We begin with a discussion of the mathematical constructs known as fields. The binary alphabet, along 
with modulo 2 addition and multiplication, is the smallest example of a field. 
\subsection{Field}
Let F be a nonempty set that is closed under two binary operations denoted +
and *. That is, the binary operations input any elements a and b in F and output other elements
c = a + b and d = a * b in F . Then (F, +, *) is a field if:

\begin{enumerate}
  \item There exists an additive identity labeled 0 in F such that a + 0 = 0 + a = a for any element a $\in$ F.
  \item The addition is commutative: a + b = b + a, for all a, b $\in$ F.
  \item The addition is associative: (a + b) + c = a + (b + c), for all a, b, c $\in$ F.
  \item There exists an additive inverse for each element: For each a $\in$ F , there exists an element denoted -a in F such that a + (-a) = 0.
  \item There exists a multiplicative identity denoted 1 in F such that a * 1 = 1 * a = a for any element a $\in$ F.
  \item The multiplication is commutative: a * b = b * a, for all a, b $\in$ F.
  \item The multiplication is associative: a * (b * c) = (a * b) * c, for all a, b, c $\in$ F.
  \item There exists a multiplicative inverse for each nonzero element: For each a 6 = 0 in F , there exists an element denoted a -1 in F such that a * a -1 = a -1 * a = 1.
  \item The operations distribute: $a * (b + c) = a * b + a * c$ and $(b + c) * a = b * a + c * a$, for all a, b, c $\in$ F.
\end{enumerate}

\subsection{Vector Spaces}
Let F be a field. A set V of elements called vectors is a vector space if for any u, v, w $\in$ V and for any c, d $\in$ F :
\begin{enumerate}
  \item u + v $\in$ V
  \item cv $\in$ V
  \item u + v = v + u
  \item u + (v + w) = (u + v) + w
  \item There exists an element denoted 0 $\in$ V such that u + 0 = u
  \item There exists an element denoted 0 $\in$ V such that u + 0 = u
  \item c(u + v) = cu = cv
  \item (c + d)u = cu + du
  \item c(du) = (cd)u
  \item 1u = u
\end{enumerate}

\subsection{Subspaces}
Let V be a vector space over a field F and let W be a subset of V . If W is a
vector space over F , then W is a vector subspace of V .

\textbf{Theorem 2.3.1.}If W is a subset of a vector space V over a field F , then W is a vector subspace
of V iff $\alpha u + \beta v \in W$ for all $\alpha$ , $\beta$, $\gamma$ $\in$ F and all u, v $\in$ W .

We can now begin to connect vector spaces to error-control codes. Earlier, we discussed
the linear code C = {000, 100, 001, 101}. We will now show that this code is a vector subspace of
the vector space V (3, 2) of binary 3-tuples. Using the above mentioned theorem in subspaces, we must check if every linear
combination of vectors in C remains in C. In fact, this is the defining
condition for a linear code. We have just discovered one of the major links between coding theory
and linear algebra: Linear codes are vector subspaces of vector spaces.

\subsection{Linear Codes}
A linear code of length n over GF (q) is a subspace of the vector space GF (q) n = V (n, q).

\textbf{Theorem 2.4.1.} Let C be a linear code. The linear combination of any set of codewords in C is a code word in C.

\textbf{Example 2.4.1} Let’s continue working with our linear code C = \{000, 100, 001, 101\}. We can see
that this code is spanned by the linearly independent vectors [100] and [001], since all other vectors
in C can be obtained as binary linear combinations of these two vectors. In other words, these two
vectors form a basis for C, so the dimension of C is 2.

\textbf{Theorem 2.4.2.}The dimension of a linear code C of length n is its dimension as a vector
subspace of $GF (2)^n = V (n, 2)$.

If C is a k-dimensional subspace of V (n, q), we say that C has dimension k and is an [n, k, d] or
[n, k] linear code. In particular, if the dimension of a binary code is k, then the number of codewords is $2^k$ . In general, if the
dimension of a q-ary code is k, then the number of codewords is $q^k$ . This means that C can be used
to communicate any of $q^k$ distinct messages. We identify these messages with the $q^k$ elements of
V (k, q), the vector space of k-tuples over GF (q). This is consistent with the aforementioned notion
of encoding messages of length k to obtain codewords of length n, where $n > k$.

\subsection{Encoding Linear Codes: Generator Matrices}
Linear codes are used in practice largely due to the simple encoding procedures facilitated by their
linearity. A k × n generator matrix G for an [n, k] linear code C provides a compact way to describe
all of the codewords in C and provides a way to encode messages. By performing the multiplication
mG, a generator matrix maps a length k message string m to a length n codeword string. The
encoding function m → mG maps the vector space V (k, q) onto a k-dimensional subspace (namely
the code C) of the vector space of V (n, q).

Before giving a formal definition of generator matrices, let’s start with some examples.

\textbf{Example 2.5.1.} We will now extend the work done in Example 2.4.1 on C = \{000, 100, 001, 101\}.
We have already shown that C is a 2-dimensional vector subspace of the vector space V (3, 2).
Therefore, C is a [3, 2] linear code and should have a 2 × 3 generator matrix. However, let’s use
our intuition to derive the size and content of the generator matrix. We want to find a generator
matrix G for C that will map messages (i.e. binary bit strings) onto the four codewords of C. But
how long are are the messages that we can encode? We can answer this question using three facts:
(1) we always encode messages of a fixed length k to codewords of a fixed length n; (2) a code must
be able to encode arbitrary messages of length k; and (3) the encoding procedure must be “one-to-
one,” or in other words, no two messages can be encoded as the same codeword. Using these three
facts, it is clear that the messages corresponding to the four codewords in C must be of length 2
since there are exactly 4 distinct binary messages of length 2: [00], [10], [01], and [11]. Hence, in
order for the matrix multiplication to make sense, G must have 2 rows. Since the codewords are of length 3, 
properties of matrix multiplication tell us that G should have 3 columns. In particular, if 
$ G =
\begin{bmatrix}
1 & 0 & 0\\
0 & 0 & 1
\end{bmatrix}
$, then
$[00]G=[000]$, $[10]G=[100]$, $[01]G=[001]$ and $[11]G=[101]$. 
Hence, this code C can encode any length 2 binary message. The generator matrix determines which length 2 message is encoded as which length 3 codeword.

We have now seen how a generator matrix can be used to map messages onto codewords. We have
also seen that an [n, k] linear code C is a k-dimensional vector subspace of V (n, q) and can hence be
specified by a basis of k codewords, or q-ary vectors. This leads us to the following formal definition
of a generator matrix: \emph{A k × n matrix G whose rows form a basis for an [n, k] linear code C is called a
generator matrix of the code C.}

\subsection{Parity Check Matrices}
Let C be an [n, k] linear code. A parity check matrix for C is an (n - k) × n
matrix H such that $c \in C$ if and only if $cH^T = 0$.

\textbf{Example 2.6.1.}If C is the code \{111, 000\}, then a valid parity check matrix is 
\[ H =
\begin{bmatrix}
1 & 0 & 1\\
0 & 1 & 1
\end{bmatrix}
\]
One way to verify this is to compute the product cH T for an arbitrary codeword c = [x, y, z]:
\[
[xyz]
\begin{bmatrix}
1 & 0 \\
0 & 1 \\
1 & 1
\end{bmatrix}
=
[x + z, y + z]
\]
So, c = [xyz] is a valid codeword iff cH T = 0, ie. iff [x + z, y + z] = [0, 0]. Algebraic manipulations
over binary show that x = y = z. Thus the only valid codewords are indeed {000, 111}. Notice that
this is the binary repetition code of length 3.
\end{document}

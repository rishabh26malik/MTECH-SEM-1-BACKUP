\documentclass[14pt, letterpaper]{article}
\usepackage[utf8]{inputenc}
\usepackage[table]{xcolor}
\usepackage[T1]{fontenc}


\begin{document}
 \textbf{GIVEN}\\
Blue dice : 2 2 2 5 5 5\\
Pink dice : 3 3 3 3 3 6\\ \\
\textbf{a)}\\ \\
P(pink beats blue upon single roll of each) = ? \\
\textbf{"Pink beats blue upon single roll of each"} means we need to count number of ways in which number on pink dice is more than the number on blue dice. \\
\\  
\textbf{This can happen in 2 ways : }\\ \\
\textbf{CASE 1 : } When 3 appears on pink dice and 2 appears on blue dice\\
P(3 on pink dice) * P(2 on blue dice) = $\frac{5}{6} * \frac{3}{6}$ \\ \\
\textbf{CASE 2 : } When 6 appears on pink dice (in this case it does not matter what number appears on blue as both 2 and 5 are less than 6)\\
P(6 on pink dice) = $\frac{1}{6}$ \\ \\
\textbf{ANSWER : }Sum of both the above cases = $\frac{5}{6} * \frac{3}{6} + \frac{1}{6} = \frac{7}{12}$
\\ \\
\textbf{b)}\\ 
The 2 tables drawn below gives the sample space of different values of sum for both the dices.\\
\textbf{Note : Blue tables represents blue dice and pink table represents pink dice}\\
%\textbf{Sum table for Blue dice}\\[+0.2cm]
    \arrayrulecolor[HTML]{0000FF}
    \begin{tabular}{|c|c|c|c|c|c|c|}
        \hline
        \cellcolor[HTML]{A1CAF1}2 & 4 & 4 & 4 & 7 & 7 & 7\\
        \hline
        \cellcolor[HTML]{A1CAF1}2 & 4 & 4 & 4 & 7 & 7 & 7\\    
        \hline
        \cellcolor[HTML]{A1CAF1}2 & 4 & 4 & 4 & 7 & 7 & 7\\ 
        \hline
        \cellcolor[HTML]{A1CAF1}5 & 7 & 7 & 7 & 10 & 10 & 10\\
        \hline
        \cellcolor[HTML]{A1CAF1}5 & 7 & 7 & 7 & 10 & 10 & 10\\
        \hline
        \cellcolor[HTML]{A1CAF1}5 & 7 & 7 & 7 & 10 & 10 & 10\\
        \hline
        \rowcolor[HTML]{A1CAF1} X & 2 & 2 & 2 & 5 & 5 & 5\\
        \hline
    \end{tabular} 
\quad
    %\textbf{Pink dice}\\[-0.1cm]
    \arrayrulecolor[HTML]{FF55A3}
    \begin{tabular}{|c|c|c|c|c|c|c|}
        \hline
        \cellcolor[HTML]{F4BBFF}3 & 6 & 6 & 6 & 6 & 6 & 9\\
        \hline
        \cellcolor[HTML]{F4BBFF}3 & 6 & 6 & 6 & 6 & 6 & 9\\    
        \hline
        \cellcolor[HTML]{F4BBFF}3 & 6 & 6 & 6 & 6 & 6 & 9\\ 
        \hline
        \cellcolor[HTML]{F4BBFF}3 & 6 & 6 & 6 & 6 & 6 & 9\\
        \hline
        \cellcolor[HTML]{F4BBFF}3 & 6 & 6 & 6 & 6 & 6 & 9\\
        \hline
        \cellcolor[HTML]{F4BBFF}6 & 9 & 9 & 9 & 9 & 9 & 12\\
        \hline
        \rowcolor[HTML]{F4BBFF} X & 3 & 3 & 3 & 3 & 3 & 6\\
        \hline
    \end{tabular} \\[+0.2cm]
%\textbf{Sum table for Blue dice} \\
\textbf{Below are the probabilities for different sum for both the dices}\\
\textbf{Blue dice}\\
$P(sum = 4)=\frac{9}{36}$,  $P(sum = 7)=\frac{18}{36}$,  $P(sum = 10)=\frac{9}{36}$ \\[+0.2cm]
\textbf{Pink dice}\\
$P(sum = 6)=\frac{25}{36}$, $P(sum = 9)=\frac{10}{36}$, $P(sum = 12)=\frac{1}{36}$ \\[+0.2cm]
P(sum of the blue dice > sum of the pink dice) can be calculated by combining 2 cases : \\
\textbf{CASE 1 : } 7 appears on blue dice and 6 appears on pink dice  \\
P(sum=7 for blue dice) * P(sum=4 for pink dice) = $\frac{18}{36} * \frac{25}{36}$ \\ \\

\textbf{CASE 2 : } 10 appears on blue dice and 6 or 9 appears on pink dice  \\
P(sum=10 for blue dice) * P(sum=4 or sum=9 for pink dice) = $\frac{9}{36} * \frac{35}{36}$ \\ \\

\textbf{ANSWER : }Sum of both the above cases = $\frac{18}{36} * \frac{25}{36} + \frac{9}{36} * \frac{35}{36} = \frac{85}{144}$
\end{document}